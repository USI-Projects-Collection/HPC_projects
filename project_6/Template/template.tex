\documentclass[unicode,11pt,a4paper,oneside,numbers=endperiod,openany]{scrartcl}

\usepackage{amsmath}
\usepackage{listings}
\usepackage{hyperref}
\usepackage{subcaption}
\usepackage{multirow}
\usepackage{float}
\usepackage{xcolor}
\usepackage{csvsimple}
\usepackage{graphicx}
\usepackage[T1]{fontenc}
\usepackage{inconsolata}

\lstdefinestyle{hpcblock}{
    basicstyle=\ttfamily\footnotesize,
    backgroundcolor=\color{gray!10},
    frame=single,
    rulecolor=\color{gray!60},
    xleftmargin=1.5em,
    xrightmargin=1.5em,
    aboveskip=1em,
    belowskip=1em,
    showstringspaces=false,
    columns=fullflexible,
    keepspaces=true,
    breaklines=true,
    postbreak=\mbox{\textcolor{gray}{$\hookrightarrow$}\space}
}
\lstset{style=hpcblock}




\input{assignment.sty}
\begin{document}


\setassignment

\serieheader{High-Performance Computing Lab}{Institute of Computing}{Student: Paolo Deidda}{Discussed with: Lino Candian}{Solution for Project 6}{}
\newline

\assignmentpolicy
\tableofcontents
\newpage

% -------------------------------------------------------------------------- %
% -------------------------------------------------------------------------- %
% --- Exercise 1 ----------------------------------------------------------- %
% -------------------------------------------------------------------------- %
% -------------------------------------------------------------------------- %
\section{Graph Partitioning with Matlab: Load balancing for HPC}

% -------------------------------------------------------------------------- %
% -------------------------------------------------------------------------- %
% --- Exercise 2 ----------------------------------------------------------- %
% -------------------------------------------------------------------------- %
% -------------------------------------------------------------------------- %
\section{Graph partitioning with Matlab: Exercises [85 points]}

% -------------------------------------------------------------------------- %
% -------------------------------------------------------------------------- %
% --- Exercise 2.1 --------------------------------------------------------- %
% -------------------------------------------------------------------------- %
% -------------------------------------------------------------------------- %
% Ex 2.1

\subsection{Task: Install METIS 5.0.2, and the corresponding Matlab mex interface}


% -------------------------------------------------------------------------- %
% -------------------------------------------------------------------------- %
% --- Exercise 2.2 --------------------------------------------------------- %
% -------------------------------------------------------------------------- %
% -------------------------------------------------------------------------- %
% Ex 2

\subsection{Task:  Construct adjacency matrices from connectivity data [10 points]}

We used the script \texttt{Source/read\_csv\_graphs.m} to handle the raw connectivity data that was given in CSV format. This dataset contains road networks from various countries, where the nodes represent intersections and the edges signify road segments.

The procedure consisted of parsing the coordinate files provided to build the list $C \in \mathbb{R}^{n \times 2}$ and the adjacency files to construct the sparse matrix $\mathbf{W} \in \mathbb{R}^{n \times n}$. A crucial part of this construction was making sure the adjacency matrix was symmetric, which is necessary for the spectral partitioning algorithms we would be using later on. In the source code, we achieved this symmetry by applying the following transformation:
\begin{equation}
    \mathbf{W} = \frac{\mathbf{W} + \mathbf{W}^T}{2}
\end{equation}
This step fixes any directional issues in the raw data, ensuring that the graph is treated as undirected.

We saved the processed sparse matrices and coordinates into binary .mat files for quicker loading in the next tasks. Then, we visualized the road networks for Vietnam and Norway using the plotting tools provided. The visualizations of these networks are displayed in Figure~\ref{fig:maps_no_vn}.

\begin{figure}[H]
    \centering
    \begin{subfigure}{0.3\textwidth}
        \centering
        \includegraphics[width=\linewidth]{../Part_Toolbox/image/ex2/VN_map.png}
        \caption{Vietnam road network}
    \end{subfigure}
    \hfill
    \begin{subfigure}{0.45\textwidth}
        \centering
        \includegraphics[width=\linewidth]{../Part_Toolbox/image/ex2/NO_map.png}
        \caption{Norway road network}
    \end{subfigure}
    \caption{Visual representation of the constructed graphs for Vietnam and Norway. The nodes are plotted according to their geographical coordinates.}
    \label{fig:maps_no_vn}
\end{figure}

% -------------------------------------------------------------------------- %
% -------------------------------------------------------------------------- %
% --- Exercise 2.3 --------------------------------------------------------- %
% -------------------------------------------------------------------------- %
% -------------------------------------------------------------------------- %
% Ex 3

\section{Task: Implement various graph partitioning algorithms [25 points]}


\begin{itemize}

\item Run in Matlab the script \texttt{Bench\_bisection.m} and familiarize yourself with the Matlab codes in the directory \texttt{Part\_Toolbox}. An overview of all functions and scripts is offered in \texttt{Contents.m}. 

\item Implement \textbf{spectral graph bisection} based on the entries of the Fiedler eigenvector. Use the incomplete Matlab file \texttt{bisection\_spectral.m} for your solution.

\item Implement \textbf{inertial graph bisection}. For a graph with 2D coordinates, this inertial bisection constructs a line such that half the nodes are on one side of the line, and half are on the other.
Use the incomplete Matlab file \texttt{bisection\_inertial.m} for your solution.
 
\item Report the bisection edgecut for all toy meshes that are either generated or loaded in the script \\ "Bench\_bisection.m." Use Table~\ref{table:bisection} to report these results.

\begin{table}[h]
\caption{Bisection results}
\centering
\begin{tabular}{l|r|r|r|r} \hline\hline 
Mesh             &  Coordinate           & Metis 5.0.2  & Spectral & Inertial  \\ \hline
mesh1e1          &   18                   &             &          &           \\             
mesh2e1          &   37                   &             &          &           \\ 
netz4504\_dual   &                        &             &          &           \\ 
stufe            &                        &             &          &           \\ 
\hline \hline
\end{tabular}
\label{table:bisection}
\end{table}

\end{itemize}

% -------------------------------------------------------------------------- %
% -------------------------------------------------------------------------- %
% --- Exercise 2.4 --------------------------------------------------------- %
% -------------------------------------------------------------------------- %
% -------------------------------------------------------------------------- %
% Ex 4


\section{Task: Recursively bisecting meshes [15 points]}

The recursive bisecion algorithm is implemented in the file \texttt{rec\_bisection.m} of the toolbox. Utilize this function within the script \texttt{Bench\_rec\_bisection.m} to recursively bisect the finite element meshes loaded within the script in 8 and 16 subgraphs. Use your inertial and spectral partitioning implementations, as well as the coordinate partitioning and the METIS bisection routine. Summarize your results in~\ref{table:Rec_bisection}. Finally, visualize the results for $p= 16$ for the case "crack".

\begin{table}[h]
\caption{Edge-cut results for recursive bi-partitioning.}
\centering
\begin{tabular}{l|r|r|r|r|r} \hline\hline 
 Case            &  Spectral             &  Metis 5.0.2    & Coordinate & Inertial  \\ \hline
 mesh3e1         &                       &                 &            &           \\             
 airfoil1        &                       &                 &            &           \\ 
 3elt            &                       &                 &            &           \\ 
 barth4          &                       &                 &            &           \\ 
 crack           &                       &                 &            &           \\ \hline \hline
\end{tabular}
\label{table:Rec_bisection}
\end{table}


\section{Task: Comparing recursive bisection to direct $k$-way partitioning [10 points]}

Use the incomplete \texttt{Bench\_metis.m} for your implementation. Compare the cut obtained from Metis 5.0.2 after applying recursive bisection and direct multiway partitioning for the graphs in question. Consult the Metis manual, and type \texttt{help metismex} in your MATLAB command line to familiarize yourself with the way the Metis recursive and direct multiway partitioning functionalities should be invoked. Summarize your results in Table~\ref{table:Compare_Metis} for 16 and 32 partitions. Comment on your results. Was this behavior anticipated? Visualize the partitioning results for both graphs for 32 partitions.

\begin{table}[h]
\caption{Comparing the number of cut edges for recursive bisection and direct multiway partitioning in Metis 5.0.2.}
\centering
\begin{tabular}{l|r|r|r|r|r|r|r|r} \hline\hline 
Partitions       &   Luxemburg           & usroads-48 &  Greece &  Switzerland &  Vietnam  &  Norway &  Russia  \\ \hline
 16              &                       &            &         &              &           &         &          \\             
 32              &                       &            &         &              &           &         &          \\ \hline \hline
\end{tabular}              
\label{table:Compare_Metis}
\end{table}

% -------------------------------------------------------------------------- %
% -------------------------------------------------------------------------- %
% --- Exercise 2.5 --------------------------------------------------------- %
% -------------------------------------------------------------------------- %
% -------------------------------------------------------------------------- %
% Ex 5

\section{Task: Utilizing graph eigenvectors [25 points]}


Provide the following illustrative results. Use the incomplete script \texttt{Bench\_eigen\_plot.m} for your implementation.

\begin{figure}[!t]
	\begin{center}
		\includegraphics[width=0.55\textwidth]{images/fiedler_airfoil.png}
		\caption{Partitioning the Airfoil graph based on the values of the Fiedler eigenvector. The two partitions are depicted in black and gray, while the cut edges in red respectively. The z-axis represents the value of the entries of the eigenvector.}
		 \label{fig:fiedler_airfoil}
	\end{center}
\end{figure}


\begin{enumerate}
    \item Plot the entries of the eigenvectors associated with the first ($\lambda_1$) and second ($\lambda_2$)  smallest eigenvalues of the graph Laplacian matrix $\mathbf{L}$ for the graph "airfoil1." Comment on the visual result. Is this behavior expected?
    \item Plot the entries of the eigenvector associated with the second smallest eigenvalue $\lambda_2$ of the Graph Laplacian matrix $\mathbf{L}$. Project each solution on the coordinate system space of the following graphs: mesh3e1, barth4, 3elt, crack. An example is shown in Figure~\ref{fig:fiedler_airfoil}, for the graph "airfoil1". \newline
    \textbf{Hint}: You might have to modify the functions \texttt{gplotg.m} and \texttt{gplotpart.m} to get the desired result.
    \item In this assignment we dealt exclusively with graphs $\mathcal{G}(V,E)$ that have coordinates associated with their nodes. This is, however, most commonly not the case when dealing with graphs, as they are in fact abstract structures, used for describing the relation $E$ over a collection of entities $V$. These entities very often cannot be described in a Euclidean coordinate space. Therefore graph drawing is a tool to visualize relational information between nodes. The optimality of graph drawing is measured in terms of computation speed the ultimate usefulness of the resulting layout~\cite{Koren05}. A successful layout should transmit the clearly the desired message, e.g the subsets of a partitioned graph.
    We will now see a spectral graph drawing method, which constructs the layout utilizing the eigenvectors of the graph Laplacian matrix $\mathbf{L}$. Draw the graphs mesh3e1, barth4, 3elt, crack, and their \textbf{spectral bi-partitioning} results using the eigenvectors to supply coordinates. Locate vertex $i$ at position:
    \begin{equation*}
        x_i = \left( \mathbf{v}_2(i), \mathbf{v}_3(i) 
        \right),
    \end{equation*}
    where $\mathbf{v}_2, \mathbf{v}_3$ are the eigenvectors associated with the 2nd and 3rd smallest eigenvalues of $\mathbf{L}$. Figure~\ref{fig:spectral_layout} illustrates these 2 ways of visualizing the partitions of the  "airfoil1" graph.
\end{enumerate}


\begin{figure}[!h]
\begin{center}
  \includegraphics[width=0.35\textwidth]{images/airfoil_part_spat.png}
  \includegraphics[width=0.35\textwidth]{images/airfoil_part_spec.png}
  \caption{Visualizing the bipartitioning of the graph "airfoil1" with 4253 nodes and 12289 edges. Left: Spatial coordinates. Right: Spectral coordinates.}
  \label{fig:spectral_layout}
\end{center}
\end{figure}
\bibliographystyle{plain}
\bibliography{template}


% -------------------------------------------------------------------------- %
% -------------------------------------------------------------------------- %
% --- Exercise 2.6 --------------------------------------------------------- %
% -------------------------------------------------------------------------- %
% -------------------------------------------------------------------------- %
% Ex 6

\subsection{Task: Utilizing graph eigenvectors [25 points]}

In this section, we analyze the spectral properties of the graph Laplacian matrix $L$ and their application to graph drawing and partitioning.

We computed the first two eigenvectors ($v_1, v_2$) associated with the smallest eigenvalues of the Laplacian for the \texttt{airfoil1} graph. The results are shown in Figure~\ref{fig:airfoil_eig}.

\begin{figure}[H]
    \centering
    \includegraphics[width=0.8\textwidth]{../Part_Toolbox/Source/Risultati_Task2_6/1_Airfoil_Eigenvectors.png}
    \caption{First and Second Eigenvectors of the \texttt{airfoil1} graph Laplacian.}
    \label{fig:airfoil_eig}
\end{figure}

\textbf{Comment on visual results:} 
This behavior is fully expected. 
\begin{itemize}
    \item The first eigenvector $v_1$ (top plot) corresponds to the eigenvalue $\lambda_1 \approx 0$ (computed as $3.14e^{-16}$). Since the graph is connected, the multiplicity of the zero eigenvalue is 1, and the associated eigenvector is constant ($v_1 = c \cdot \mathbf{1}$).
    \item The second eigenvector $v_2$ (bottom plot) is the \textit{Fiedler Vector}. It is orthogonal to $v_1$ and its values vary smoothly across the graph indices, crossing zero. This variation provides the heuristic for spectral bisection: nodes with $v_2 < 0$ belong to one partition, and nodes with $v_2 \ge 0$ to the other.
\end{itemize}

\subsection{Fiedler Vector Projection (Physical Space)}
We visualized the values of the Fiedler vector ($v_2$) by projecting them onto the physical coordinate system of four different meshes. The value of $v_2$ is represented by both the Z-axis height and the color map.

\begin{figure}[H]
    \centering
    \begin{subfigure}{0.45\textwidth}
        \includegraphics[width=\linewidth]{../Part_Toolbox/Source/Risultati_Task2_6/mesh3e1_Physical_3D.png}
        \caption{mesh3e1}
    \end{subfigure}
    \begin{subfigure}{0.45\textwidth}
        \includegraphics[width=\linewidth]{../Part_Toolbox/Source/Risultati_Task2_6/crack_Physical_3D.png}
        \caption{crack}
    \end{subfigure}
    \caption{Projection of the Fiedler vector ($v_2$) values onto the physical coordinates of the meshes. The color gradient shows how the spectral value varies smoothly across the geometry, identifying the optimal cut.}
    \label{fig:physical_3d}
\end{figure}

\subsection{Spectral Graph Drawing}
Finally, we performed spectral graph drawing by utilizing the second ($v_2$) and third ($v_3$) eigenvectors as Cartesian coordinates ($x, y$) for the nodes. The results for \texttt{3elt} and \texttt{barth4} are shown below.

\begin{figure}[H]
    \centering
    \begin{subfigure}{0.45\textwidth}
        \includegraphics[width=\linewidth]{../Part_Toolbox/Source/Risultati_Task2_6/3elt_Spectral.png}
        \caption{3elt Spectral Layout}
    \end{subfigure}
    \begin{subfigure}{0.45\textwidth}
        \includegraphics[width=\linewidth]{../Part_Toolbox/Source/Risultati_Task2_6/barth4_Spectral.png}
        \caption{barth4 Spectral Layout}
    \end{subfigure}
    \caption{Spectral Graph Drawing using coordinates $(v_2, v_3)$. The colors (Cyan/Yellow) represent the bisection cut. In this spectral space, nodes are clustered by connectivity rather than physical distance, making the partition boundary appear as a clear line.}
    \label{fig:spectral_drawing}
\end{figure}

The spectral layout reveals the intrinsic topology of the graphs. Nodes that are highly connected are placed close together in this space, "unfolding" the graph such that the bisection cut appears geometrically simple.

% -------------------------------------------------------------------------- %
% -------------------------------------------------------------------------- %
% --- Report Quality ------------------------------------------------------- %
% -------------------------------------------------------------------------- %
% -------------------------------------------------------------------------- %
\section{Task: Quality of the Report [15 Points] }


\section*{Additional notes and submission details}
Submit the source code files (together with your used \texttt{Makefile}) in
an archive file (tar, zip, etc.), and summarize your results and the
observations for all exercises by writing an extended Latex report.
Use the Latex template from the webpage and upload the Latex summary
as a PDF to \href{https://www.icorsi.ch}{iCorsi}.

\begin{itemize}
	\item Your submission should be a gzipped tar archive, formatted like project\_number\_lastname\_firstname.zip or project\_number\_lastname\_firstname.tgz. 
	It should contain:
	\begin{itemize}
		\item all the source codes of your MATLAB solutions;
		\item your write-up with your name  project\_number\_lastname\_firstname.pdf.
	\end{itemize}
	\item Submit your .zip/.tgz through Icorsi.
\end{itemize}

\end{document}
