% Ex 3

\section{Task: Implement various graph partitioning algorithms [25 points]}


\begin{itemize}

\item Run in Matlab the script \texttt{Bench\_bisection.m} and familiarize yourself with the Matlab codes in the directory \texttt{Part\_Toolbox}. An overview of all functions and scripts is offered in \texttt{Contents.m}. 

\item Implement \textbf{spectral graph bisection} based on the entries of the Fiedler eigenvector. Use the incomplete Matlab file \texttt{bisection\_spectral.m} for your solution.

\item Implement \textbf{inertial graph bisection}. For a graph with 2D coordinates, this inertial bisection constructs a line such that half the nodes are on one side of the line, and half are on the other.
Use the incomplete Matlab file \texttt{bisection\_inertial.m} for your solution.
 
\item Report the bisection edgecut for all toy meshes that are either generated or loaded in the script \\ "Bench\_bisection.m." Use Table~\ref{table:bisection} to report these results.

\begin{table}[h]
\caption{Bisection results}
\centering
\begin{tabular}{l|r|r|r|r} \hline\hline 
Mesh             &  Coordinate           & Metis 5.0.2  & Spectral & Inertial  \\ \hline
mesh1e1          &   18                   &             &          &           \\             
mesh2e1          &   37                   &             &          &           \\ 
netz4504\_dual   &                        &             &          &           \\ 
stufe            &                        &             &          &           \\ 
\hline \hline
\end{tabular}
\label{table:bisection}
\end{table}

\end{itemize}