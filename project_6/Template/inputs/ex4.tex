% Ex 4


\subsection{Task: Recursively bisecting meshes [15 points]}

The recursive bisecion algorithm is implemented in the file \texttt{rec\_bisection.m} of the toolbox. Utilize this function within the script \texttt{Bench\_rec\_bisection.m} to recursively bisect the finite element meshes loaded within the script in 8 and 16 subgraphs. Use your inertial and spectral partitioning implementations, as well as the coordinate partitioning and the METIS bisection routine. Summarize your results in~\ref{table:Rec_bisection}. Finally, visualize the results for $p= 16$ for the case "crack".

\begin{table}[h]
\caption{Edge-cut results for recursive bi-partitioning.}
\centering
\begin{tabular}{l|r|r|r|r|r} \hline\hline 
 Case            &  Spectral             &  Metis 5.0.2    & Coordinate & Inertial  \\ \hline
 mesh3e1         &                       &                 &            &           \\             
 airfoil1        &                       &                 &            &           \\ 
 3elt            &                       &                 &            &           \\ 
 barth4          &                       &                 &            &           \\ 
 crack           &                       &                 &            &           \\ \hline \hline
\end{tabular}
\label{table:Rec_bisection}
\end{table}
