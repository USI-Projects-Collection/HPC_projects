% Ex 2.4


\subsection{Task: Recursively bisecting meshes [15 points]}

The recursive bisection algorithm is implemented in \texttt{rec\_bisection.m}.  
Using the script \texttt{Bench\_rec\_bisection.m}, each mesh is recursively partitioned into 8 and 16 subgraphs.  
At each level of recursion, the chosen bisection method comes into play, helping us see how different strategies like spectral, Metis, coordinate-based, and inertial affect the overall edge-cut during the hierarchical breakdown. You can find a summary of the edge-cut values for all scenarios in Table~\ref{table:Rec_bisection}.  
The visual inspection requested for the case \emph{crack} with \(p = 16\) is also performed within the same script.

\begin{table}[h]
\caption{Edge-cut results for recursive bi-partitioning.}
\centering
\begin{tabular}{l|rr|rr|rr|rr} \hline\hline
\multirow{2}{*}{Graph} & \multicolumn{2}{c|}{Spectral} & \multicolumn{2}{c|}{Metis 5.0.2} & \multicolumn{2}{c|}{Coordinate} & \multicolumn{2}{c}{Inertial} \\ 
 & 8 & 16 & 8 & 16 & 8 & 16 & 8 & 16 \\ \hline
airfoil1        & 397 & 631 & 318 & 561 & 516 & 819 & 1276 & 2531 \\
netz4504\_dual  & 111 & 185 & 96  & 159 & 127 & 198 & 237  & 483  \\
stufe           & 128 & 243 & 108 & 193 & 123 & 227 & 265  & 495  \\
3elt            & 469 & 752 & 418 & 699 & 733 & 1168 & 881 & 1608 \\
barth4          & 549 & 835 & 470 & 743 & 875 & 1306 & 1299 & 2353 \\
ukerbe1         & 126 & 236 & 147 & 245 & 225 & 374 & 435 & 785 \\
crack           & 883 & 1419 & 808 & 1275 & 1343 & 1860 & 1706 & 3447 \\ \hline\hline
\end{tabular}
\label{table:Rec_bisection}
\end{table}
