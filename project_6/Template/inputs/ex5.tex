% Ex 5

\section{Task: Utilizing graph eigenvectors [25 points]}


Provide the following illustrative results. Use the incomplete script \texttt{Bench\_eigen\_plot.m} for your implementation.

\begin{figure}[!t]
	\begin{center}
		\includegraphics[width=0.55\textwidth]{images/fiedler_airfoil.png}
		\caption{Partitioning the Airfoil graph based on the values of the Fiedler eigenvector. The two partitions are depicted in black and gray, while the cut edges in red respectively. The z-axis represents the value of the entries of the eigenvector.}
		 \label{fig:fiedler_airfoil}
	\end{center}
\end{figure}


\begin{enumerate}
    \item Plot the entries of the eigenvectors associated with the first ($\lambda_1$) and second ($\lambda_2$)  smallest eigenvalues of the graph Laplacian matrix $\mathbf{L}$ for the graph "airfoil1." Comment on the visual result. Is this behavior expected?
    \item Plot the entries of the eigenvector associated with the second smallest eigenvalue $\lambda_2$ of the Graph Laplacian matrix $\mathbf{L}$. Project each solution on the coordinate system space of the following graphs: mesh3e1, barth4, 3elt, crack. An example is shown in Figure~\ref{fig:fiedler_airfoil}, for the graph "airfoil1". \newline
    \textbf{Hint}: You might have to modify the functions \texttt{gplotg.m} and \texttt{gplotpart.m} to get the desired result.
    \item In this assignment we dealt exclusively with graphs $\mathcal{G}(V,E)$ that have coordinates associated with their nodes. This is, however, most commonly not the case when dealing with graphs, as they are in fact abstract structures, used for describing the relation $E$ over a collection of entities $V$. These entities very often cannot be described in a Euclidean coordinate space. Therefore graph drawing is a tool to visualize relational information between nodes. The optimality of graph drawing is measured in terms of computation speed the ultimate usefulness of the resulting layout~\cite{Koren05}. A successful layout should transmit the clearly the desired message, e.g the subsets of a partitioned graph.
    We will now see a spectral graph drawing method, which constructs the layout utilizing the eigenvectors of the graph Laplacian matrix $\mathbf{L}$. Draw the graphs mesh3e1, barth4, 3elt, crack, and their \textbf{spectral bi-partitioning} results using the eigenvectors to supply coordinates. Locate vertex $i$ at position:
    \begin{equation*}
        x_i = \left( \mathbf{v}_2(i), \mathbf{v}_3(i) 
        \right),
    \end{equation*}
    where $\mathbf{v}_2, \mathbf{v}_3$ are the eigenvectors associated with the 2nd and 3rd smallest eigenvalues of $\mathbf{L}$. Figure~\ref{fig:spectral_layout} illustrates these 2 ways of visualizing the partitions of the  "airfoil1" graph.
\end{enumerate}


\begin{figure}[!h]
\begin{center}
  \includegraphics[width=0.35\textwidth]{images/airfoil_part_spat.png}
  \includegraphics[width=0.35\textwidth]{images/airfoil_part_spec.png}
  \caption{Visualizing the bipartitioning of the graph "airfoil1" with 4253 nodes and 12289 edges. Left: Spatial coordinates. Right: Spectral coordinates.}
  \label{fig:spectral_layout}
\end{center}
\end{figure}
\bibliographystyle{plain}
\bibliography{template}
