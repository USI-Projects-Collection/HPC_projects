\documentclass[unicode,11pt,a4paper,oneside,numbers=endperiod,openany]{scrartcl}

\input{assignment.sty}

\usepackage{listings}
\usepackage{hyperref}
\usepackage{subcaption}
\usepackage{multirow}
\usepackage{float}
\usepackage{xcolor}
\usepackage{csvsimple}
\usepackage{graphicx}
\usepackage[T1]{fontenc}
\usepackage{inconsolata}


\lstdefinestyle{hpcblock}{
    basicstyle=\ttfamily\footnotesize,
    backgroundcolor=\color{gray!10},
    frame=single,
    rulecolor=\color{gray!60},
    xleftmargin=1.5em,
    xrightmargin=1.5em,
    aboveskip=1em,
    belowskip=1em,
    showstringspaces=false,
    columns=fullflexible,
    keepspaces=true,
    breaklines=true,
    postbreak=\mbox{\textcolor{gray}{$\hookrightarrow$}\space}
}
\lstset{style=hpcblock}

\begin{document}


\setassignment

\serieheader{High-Performance Computing Lab}{Institute of Computing}{Student: Paolo Deidda}{Discussed with: FULL NAME}{Solution for Project 4}{}
\newline

\assignmentpolicy

\tableofcontents
\newpage

% -------------------------------------------------------------------------- %
% -------------------------------------------------------------------------- %
% --- Exercise 1 ----------------------------------------------------------- %
% -------------------------------------------------------------------------- %
% -------------------------------------------------------------------------- %

\section{Task: Ring maximum using MPI [10 Points]}

I detrmine each process message destination and source using its rank.

The communication can be implemented using the standard MPI functions 
\texttt{MPI\_Send} and \texttt{MPI\_Recv}, which take the following arguments:
\begin{itemize}
    \setlength{\itemsep}{0pt}
    \setlength{\parskip}{0pt}
    \item buffer address (to store sent or received data)
    \item number of elements (to send or receive)
    \item MPI datatype (of the elements to send or receive)
    \item destination (or source) rank (of the process to send to or receive from)
    \item message tag (to identify the message with an id)
    \item communicator (to identify the group of processes involved in the communication, useless in this case)
\end{itemize}

When using \texttt{MPI\_Send} and \texttt{MPI\_Recv} separately, the ranks must be split into two groups like even and odd ranks, otherwise all processes may block by attempting to send or receive at the same time.

Alternatively, we can use the \texttt{MPI\_Sendrecv} function, which takes care of both sending and receiving in a single call, avoiding deadlocks.
\lstinputlisting[
    language=C++,
    numbers=left,
    caption={Implemetation from file \href{../Skeleton_codes/ring/ring_sum.c}{ring\_sum.c}},
    captionpos=b,
    label={lst:mergeCritical},
    linerange={21-25},
    firstnumber=21
]{../Skeleton_codes/ring/ring_sum.c}


 % -------------------------------------------------------------------------- %
% -------------------------------------------------------------------------- %
% --- Exercise 2 ----------------------------------------------------------- %
% -------------------------------------------------------------------------- %
% -------------------------------------------------------------------------- %

\section{Task: Ghost cells exchange between neighboring processes [15 Points]}

We first create a $4\times 4$ Cartesian communicator with periodic boundaries in both directions. Using \texttt{MPI\_Cart\_shift} we retrieve the rank of the north, south, east and west neighbors so that the first and last ranks remain connected. The column ghost layer is described with a derived datatype built with \texttt{MPI\_Type\_vector}; it spans one element every \texttt{DOMAINSIZE} entries and covers the whole interior height of the tile.

With these ingredients we perform four \texttt{MPI\_Sendrecv} calls, one per direction, each time sending the interior boundary (rows or columns that exclude the corner points) and receiving directly into the ghost layer. Because every transfer is expressed as a combined send/receive, no explicit ordering or buffering is needed to avoid deadlocks.
\lstinputlisting[
    language=C,
    numbers=left,
    caption={Ghost layer exchange, \href{../Skeleton_codes/ghost/ghost.c}{ghost.c}},
    captionpos=b,
    label={lst:ghostExchange},
    linerange={80-140},
    firstnumber=80
]{../Skeleton_codes/ghost/ghost.c}

% -------------------------------------------------------------------------- %
% -------------------------------------------------------------------------- %
% --- Exercise 3 ----------------------------------------------------------- %
% -------------------------------------------------------------------------- %
% -------------------------------------------------------------------------- %

\section{Task: Parallelizing the Mandelbrot set using MPI [20 Points]}



% -------------------------------------------------------------------------- %
% -------------------------------------------------------------------------- %
% --- Exercise 4 ----------------------------------------------------------- %
% -------------------------------------------------------------------------- %
% -------------------------------------------------------------------------- %

\section{Task: Parallel matrix-vector multiplication and the power method [40 Points]}

% -------------------------------------------------------------------------- %
% -------------------------------------------------------------------------- %
% --- Report Quality ------------------------------------------------------- %
% -------------------------------------------------------------------------- %
% -------------------------------------------------------------------------- %


\section{Task:  Quality of the Report [15 Points]}


\end{document}
