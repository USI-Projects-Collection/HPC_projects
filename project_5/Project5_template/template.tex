\documentclass[unicode,11pt,a4paper,oneside,numbers=endperiod,openany]{scrartcl}

\usepackage{listings}
\usepackage{hyperref}
\usepackage{subcaption}
\usepackage{multirow}
\usepackage{float}
\usepackage{xcolor}
\usepackage{csvsimple}
\usepackage{graphicx}
\usepackage{amsmath}
\usepackage[T1]{fontenc}
\usepackage{inconsolata}

\lstdefinestyle{hpcblock}{
    basicstyle=\ttfamily\footnotesize,
    backgroundcolor=\color{gray!10},
    frame=single,
    rulecolor=\color{gray!60},
    xleftmargin=1.5em,
    xrightmargin=1.5em,
    aboveskip=1em,
    belowskip=1em,
    showstringspaces=false,
    columns=fullflexible,
    keepspaces=true,
    breaklines=true,
    postbreak=\mbox{\textcolor{gray}{$\hookrightarrow$}\space}
}
\lstset{style=hpcblock}


\input{assignment.sty}
\begin{document}


\setassignment

\serieheader{High-Performance Computing Lab}{Institute of Computing}{Student: Paolo Deidda}{Discussed with: FULL NAME}{Solution for Project 5}{}
\newline

\assignmentpolicy

\tableofcontents
\newpage

% =============================================================================
\section{Parallel Space Solution of a nonlinear PDE using MPI [in total 60 points]}
% =============================================================================

\subsection{Initialize/finalize MPI and welcome message [5 Points]}
Output:
\lstinputlisting[
    language=C++,
    % numbers=left,
    % caption={Welcome message update in main.cpp},
    captionpos=b,
    label={lst:mergeCritical},
    % linerange={0},
    % firstnumber=111
]{../Skeleton_codes/mini_app/out_ex1_1.txt}

\subsection{Domain decomposition [10 Points]}

I use a two–dimensional Cartesian domain decomposition:

\begin{itemize}
    \item \textbf{Process grid creation.}  
    I obtain a balanced $d_x \times d_y$ grid using \texttt{MPI\_Dims\_create}, which guarantees that the process grid is as close to square as possible for any number of processes~$P$.

    \item \textbf{Cartesian topology.}  
    I build the logical topology with \texttt{MPI\_Cart\_create}, so that each process can identify its coordinates and its four neighbors through \texttt{MPI\_Cart\_coords} and \texttt{MPI\_Cart\_shift}.

    \item \textbf{Subdomain sizes.}  
    The global domain $n \times n$ is partitioned into rectangular subdomains.  
    Local sizes $(n_x, n_y)$ are computed from $(d_x, d_y)$, and any remainder is distributed across the first processes in each direction so that subdomain sizes differ by at most one cell.

    \item \textbf{Load balancing.}  
    With this decomposition all processes receive almost the same number of grid points $N = n_x n_y$.

    \item \textbf{Communication overhead.}  
    In a 2D layout, the amount of halo data exchanged with neighbors grows only as $\mathcal{O}(n_x + n_y)$, while work grows as $\mathcal{O}(n_x n_y)$.  
    This results in a lower communication-to-computation ratio than a 1D decomposition.
\end{itemize}

\subsection{Linear algebra kernels [5 Points]}

Only two kernels required MPI parallelization:

\begin{itemize}
    \item \textbf{hpc\_dot}: computes a global inner product.  
    I added \texttt{MPI\_Allreduce} to combine the partial sums from all ranks.

    \item \textbf{hpc\_norm2}: computes the global 2-norm.  
    I used \texttt{MPI\_Allreduce} to sum the local squared values before taking the square root.
\end{itemize}

All other \texttt{hpc\_xxx} functions operate purely on local data (vector updates, axpy, copy, scale) and therefore do not require communication or any MPI modification.

\subsection{The diffusion stencil: Ghost cells exchange [10 Points]}

I implemented the ghost-cell exchange using \textbf{non-blocking point-to-point communication} so that communication and computation can overlap.  
The steps and MPI calls used are:

\begin{itemize}
    \item \textbf{Packing send buffers.}  
    Before communication, each process copies its local boundary data into four buffers: \texttt{buffN}, \texttt{buffS}, \texttt{buffE}, \texttt{buffW}.

    \item \textbf{Posting non-blocking receives: \texttt{MPI\_Irecv}.}  
    For every existing neighbor (north, south, east, west), I post an \texttt{MPI\_Irecv} on the corresponding receive buffer (\texttt{bndN}, \texttt{bndS}, \texttt{bndE}, \texttt{bndW}).  
    Since \texttt{MPI\_Irecv} is non-blocking, control returns immediately without waiting for the message.

    \item \textbf{Posting non-blocking sends: \texttt{MPI\_Isend}.}  
    After the receives, each process posts the matching \texttt{MPI\_Isend} using the packed buffers.  
    Being non-blocking, these sends also do not stop execution.

    \item \textbf{Overlap of communication and computation.}  
    Thanks to the non-blocking operations, the stencil computation on all \textbf{interior points} (i.e.\ grid points that do not require ghost data) can proceed while data is in transit.  
    Only boundary points depend on the incoming ghost values, so they are computed later.

    \item \textbf{Waiting for completion: \texttt{MPI\_Waitall}.}  
    After finishing the interior region, all pending communications are completed with a single \texttt{MPI\_Waitall}.  
    Once the ghost buffers are updated, the stencil is applied safely to the boundary and corner points.
\end{itemize}

A blocking version (\texttt{MPI\_Send}, \texttt{MPI\_Recv}, or \texttt{MPI\_Sendrecv}) would force each process to wait for neighbors before continuing, preventing overlap and reducing performance.  
Using \texttt{MPI\_Irecv}, \texttt{MPI\_Isend}, and \texttt{MPI\_Waitall} allows the solver to hide communication costs behind the interior computation, improving overall efficiency.


\subsection{Implement parallel I/O [10 Points]}

I implemented the final output using \textbf{MPI-IO} so that all processes write
their local subdomains into a single global binary file.  
The steps are:

\begin{itemize}
    \item \textbf{MPI\_File\_open:}  
    The file is opened collectively by all ranks in write-only mode.

    \item \textbf{Offset computation:}  
    Each process computes its byte offset inside the global $n \times n$ array
    using its global start indices $(\text{start}_x, \text{start}_y)$ provided by
    the domain decomposition.

    \item \textbf{Derived datatype:}  
    I created a derived MPI datatype using \texttt{MPI\_Type\_contiguous} to
    represent a full local row of $n_x$ doubles, simplifying the write calls.

    \item \textbf{MPI\_File\_write\_at:}  
    Each process writes its $n_y$ local rows into the correct position in the
    global file without any file locking or coordination between processes.

    \item \textbf{Metadata:}  
    Only rank 0 writes the accompanying \texttt{.bov} visualization header.
\end{itemize}

This method produces a single binary array that contains the full global
solution, with no need for post-processing or intermediate files.

\subsection{Strong scaling [10 Points]}

\begin{figure}[H]
    \centering
    \begin{subfigure}[b]{0.48\textwidth}
        \centering
        \includegraphics[width=\textwidth]{../../project_3/Skeleton_codes/mini_app/strong_scaling/strong_scaling_time_vs_threads.png}
        \caption{strong scaling time vs threads with openMP}
        \label{fig:lowres}
    \end{subfigure}
    \hfill
    \begin{subfigure}[b]{0.48\textwidth}
        \centering
        \includegraphics[width=\textwidth]{../../project_3/Skeleton_codes/mini_app/strong_scaling/strong_scaling_speedup.png}
        \caption{strong scaling speedup with openMP}
        \label{fig:highres}
    \end{subfigure}
    \begin{subfigure}[b]{0.48\textwidth}
        \centering
        \includegraphics[width=\textwidth]{../Skeleton_codes/mini_app/strong_scaling/strong_scaling_time_vs_threads.png}
        \caption{strong scaling time vs threads without MPI}
        \label{fig:lowres}
    \end{subfigure}
    \hfill
    \begin{subfigure}[b]{0.48\textwidth}
        \centering
        \includegraphics[width=\textwidth]{../Skeleton_codes/mini_app/strong_scaling/strong_scaling_speedup.png}
        \caption{}{strong scaling speedup without MPI}
        \label{fig:highres}
    \end{subfigure}
    \caption{Above results with OpenMP (from project 3), below results without MPI.}
    \label{fig:resolutions}
\end{figure}

\subsection{Weak scaling [10 Points]}


\begin{figure}[H]
    \centering
    \begin{subfigure}[b]{0.48\textwidth}
        \centering
        \includegraphics[width=\textwidth]{../../project_3/Skeleton_codes/mini_app/weak_scaling/weak_scaling_time_vs_threads.png}
        \caption{weak scaling time vs threads with openMP}
        \label{fig:lowres}
    \end{subfigure}
    \hfill
    \begin{subfigure}[b]{0.48\textwidth}
        \centering
        \includegraphics[width=\textwidth]{../../project_3/Skeleton_codes/mini_app/weak_scaling/weak_scaling_efficiency.png}
        \caption{weak scaling speedup with openMP}
        \label{fig:highres}
    \end{subfigure}
    \begin{subfigure}[b]{0.48\textwidth}
        \centering
        \includegraphics[width=\textwidth]{../Skeleton_codes/mini_app/weak_scaling/weak_scaling_time_vs_threads.png}
        \caption{strong scaling time vs threads without MPI}
        \label{fig:lowres}
    \end{subfigure}
    \hfill
    \begin{subfigure}[b]{0.48\textwidth}
        \centering
        \includegraphics[width=\textwidth]{../Skeleton_codes/mini_app/weak_scaling/weak_scaling_efficiency.png}
        \caption{weak scaling efficiency without MPI}
        \label{fig:highres}
    \end{subfigure}
    \caption{Above results with OpenMP (from project 3), below results without MPI.}
    \label{fig:resolutions}
\end{figure}
% =============================================================================
\section{Python for High-Performance Computing [in total 25 points]}
% =============================================================================

\subsection{Sum of ranks: MPI collectives [5 Points]}

\subsection{Ghost cell exchange between neighboring processes [5 Points]}

\subsection{A self-scheduling example: Parallel Mandelbrot [15 Points]}

% =============================================================================
\section{Task:  Quality of the Report [15 Points]}
% =============================================================================




\end{document}
