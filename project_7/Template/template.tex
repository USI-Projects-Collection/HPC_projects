\documentclass[unicode,11pt,a4paper,oneside,numbers=endperiod,openany]{scrartcl}

\usepackage{amsmath}
\usepackage{listings}
\usepackage{hyperref}
\usepackage{subcaption}
\usepackage{multirow}
\usepackage{float}
\usepackage{xcolor}
\usepackage{csvsimple}
\usepackage{graphicx}
\usepackage[T1]{fontenc}
\usepackage{inconsolata}
\usepackage{amsmath}
\usepackage{amssymb}

\lstdefinestyle{hpcblock}{
    basicstyle=\ttfamily\footnotesize,
    backgroundcolor=\color{gray!10},
    frame=single,
    rulecolor=\color{gray!60},
    xleftmargin=1.5em,
    xrightmargin=1.5em,
    aboveskip=1em,
    belowskip=1em,
    showstringspaces=false,
    columns=fullflexible,
    keepspaces=true,
    breaklines=true,
    postbreak=\mbox{\textcolor{gray}{$\hookrightarrow$}\space}
}
\lstset{style=hpcblock}



\input{assignment.sty}
\begin{document}


\setassignment

\serieheader{High-Performance Computing Lab}{Institute of Computing}{Student: Paolo Deidda}{Discussed with: FULL NAME}{Solution for Project 7}{}
\newline

\assignmentpolicy

\tableofcontents
\newpage

% -------------------------------------------------------------------------- %
% -------------------------------------------------------------------------- %
% --- Exercise 1 ----------------------------------------------------------- %
% -------------------------------------------------------------------------- %
% -------------------------------------------------------------------------- %

\section{HPC Mathematical Software for Extreme-Scale Science  [85 points]}

\subsection{Boundary problem above in Python [25 points]}

\subsection{Boundary problem above in PETSc [25 points]}

\subsection{Validate and Visualize [10 points]}

\subsection{Performance Benchmark [15 points]}
                     
\subsection{Strong Scaling [10 points]}

% -------------------------------------------------------------------------- %
% -------------------------------------------------------------------------- %
% --- Report Quality ------------------------------------------------------- %
% -------------------------------------------------------------------------- %
% -------------------------------------------------------------------------- %
\newpage
\section{Task: Quality of the Report [15 Points] }


\section*{Additional notes and submission details}
Submit the source code files (together with your used \texttt{Makefile}) in
an archive file (tar, zip, etc.), and summarize your results and the
observations for all exercises by writing an extended Latex report.
Use the Latex template from the webpage and upload the Latex summary
as a PDF to \href{https://www.icorsi.ch}{iCorsi}.

\begin{itemize}
	\item Your submission should be a gzipped tar archive, formatted like project\_number\_lastname\_firstname.zip or project\_number\_lastname\_firstname.tgz. 
	It should contain:
	\begin{itemize}
		\item all the source codes of your solutions;
		\item your write-up with your name  project\_number\_lastname\_firstname.pdf.
	\end{itemize}
	\item Submit your .zip/.tgz through Icorsi.
\end{itemize}

\end{document}
